\documentclass[nobib]{tufte-handout}
\usepackage[utf8]{inputenc}

%\geometry{showframe} % display margins for debugging page layout

\usepackage{graphicx} % allow embedded images
  \setkeys{Gin}{width=\linewidth,totalheight=\textheight,keepaspectratio}
  \graphicspath{{graphics/}} % set of paths to search for images
\usepackage{latexsym,amsmath,amssymb}  % extended mathematics
\usepackage{booktabs} % book-quality tables
\usepackage{units}    % non-stacked fractions and better unit spacing
\usepackage{multicol} % multiple column layout facilities
\usepackage{lipsum}   % filler text
\usepackage{fancyvrb} % extended verbatim environments
  \fvset{fontsize=\normalsize}% default font size for fancy-verbatim environments

\usepackage{marginfix}

\usepackage{hyphenat}
\usepackage[backend=biber,
            hyperref=true,
            autolang=hyphen,
            style=authortitle,
            citestyle=verbose,
            bibencoding=UTF-8,
            giveninits=true,
            maxbibnames=1000,
            terseinits=true,
            block=none,
            sorting=nyt,
            autocite=footnote
]{biblatex}
\addbibresource{refs.bib}
%\renewcommand{\cite}[2][0pt]{\sidenote[][#1]{\fullcite{#2}}}

% remove "in: " from articles
\renewbibmacro{in:}{%
  \ifentrytype{article}{}{%
  \printtext{\bibstring{in}\intitlepunct}}
}

%better printing of numbers
\usepackage[T1]{fontenc}
\usepackage[english]{babel}
\usepackage{csquotes}
\usepackage{textcomp}

\usepackage{ccicons}
\usepackage{fontawesome}
\usepackage{calc}
\newlength\myheight
\newlength\mydepth
\settototalheight\myheight{Xygp}
\settodepth\mydepth{Xygp}
\setlength\fboxsep{0pt}
\newcommand*\inlinegraphics[1]{%
  \settototalheight\myheight{Xygp}%
  \settodepth\mydepth{Xygp}%
  \raisebox{-\mydepth}{\includegraphics[height=\myheight]{#1}}%
}

% Standardize command font styles and environments
\newcommand{\doccmd}[1]{\texttt{\textbackslash#1}}% command name -- adds backslash automatically
\newcommand{\docopt}[1]{\ensuremath{\langle}\textrm{\textit{#1}}\ensuremath{\rangle}}% optional command argument
\newcommand{\docarg}[1]{\textrm{\textit{#1}}}% (required) command argument
\newcommand{\docenv}[1]{\textsf{#1}}% environment name
\newcommand{\docpkg}[1]{\texttt{#1}}% package name
\newcommand{\doccls}[1]{\texttt{#1}}% document class name
\newcommand{\docclsopt}[1]{\texttt{#1}}% document class option name
\newenvironment{docspec}{\begin{quote}\noindent}{\end{quote}}% command specification environment


\title{Open science and the future university researcher\thanks{This
work is licensed under a
\href{https://creativecommons.org/licenses/by/4.0/}{Creative Commons Attribution
4.0 International License} \ccby}}

\author[Kyle E.~Niemeyer]{Kyle E.~Niemeyer\thanks{
School of Mechanical, Industrial, \& Manufacturing Engineering,\\
\noindent Oregon State University\\
\noindent \href{mailto:kyle.niemeyer@oregonstate.edu}{\faEnvelopeO{}~kyle.niemeyer@oregonstate.edu}\\
\noindent \href{https://orcid.org/}{\inlinegraphics{orcid_128x128.png}}~{\scriptsize\href{https://orcid.org/0000-0003-4425-7097}{https://orcid.org/0000-0003-4425-7097}}
}}

\date{5 February 2017} % without \date command, current date is supplied


\begin{document}

\maketitle% this prints the handout title, author, and date

\begin{abstract}
\noindent
This white paper was written as a contribution to the ``Imagining Tomorrow's University:
Rethinking scholarship, education, and institutions for an open, networked era''
workshop, a joint NIH\slash NSF-funded event held 8--9 March 2017 in Rosemont, IL.
In this paper, I present an overview of what I consider open science,
why it is important, and how it plays a role in my research agenda.
I also discuss challenges faced in pursuing research openness.
This views are based on my experiences and plans. As such, all views expressed
here are personal, and do not necessarily represent those
of my institution or colleagues.
\end{abstract}

\section{Introduction}
\label{sec:intro}

Working openly should be the default mode of science---after all, how can
we advance knowledge ``by standing on the shoulders of giants''\footnote{``If I
have seen further, it is by standing on the shoulders of giants.'' Isaac
Newton (1676), although similar statements can be found as far back as the 12th
century.} if we cannot access or see those shoulders? This paper operates
on the following definition of open science:
\begin{quote}
Open science, or more broadly open research, describes the activity of performing
scientific research in a manner that makes products and findings accessible to
anyone. This includes sharing data openly (open data), publicly releasing the
source code for research software (open-source software), and making the written
products of research openly accessible (open access).
\end{quote}
In this paper, I describe my views on why open science practices are important
(supported by relevant studies), and discuss the potential societal impact of
increased openness. I also summarize my policy towards open science in my
research agenda. Finally, I call out pressing challenges to increased openness
in research, and lay out four recommended changes for university leaders that
could help encourage faculty to adopt better open practices in their research.


\section{Importance of open science}
\label{sec:importance}

Transforming research communities from traditional, closed environments to open
ones is important for a number of reasons, including (but not necessarily limited
to) the following six. McKiernan et al.\autocite{McKiernan:2016iz} discuss these
and additional benefits for researchers working openly. Tennant et al.\autocite{Tennant:2016bi}
review in detail the benefits of open-access publications.

\paragraph{Accessibility:}
Openness in research ensures that research products,
particularly written output, remain accessible to the public. This includes the
research community, funders, policy makers, and the general public. Accessibility
of research products is particularly important for publicly funded
research---since the public paid for the research, the public should have access
and be able to benefit from it.
(This does not prevent innovators or other parties from developing commercial
intellectual property based on the findings, but ensures that the original
discovery, when funded by the public, remains accessible to all.)

\paragraph{Reproducibility:}
Releasing products of research, including software and data, helps enable
reproducibility. This is particularly true for computational science, where a
written description of methods can never describe an approach as completely as
source code. In general, access to research software used to perform a
computational study, or the data from an experimental study, should enable
others to reproduce the findings of the original researchers.
However, open science is a necessary but not sufficient aspect of reproducibility,
as it can be challenging to reproduce\slash replicate results even with
available research software and data\autocite{Mesnard:2016,Barba:2016ky}.

\paragraph{Impact:}
As a selfish motivation, performing research openly helps increase the impact of work.
Studies have shown that open-access papers are cited more in most research fields.
In engineering, open-access papers are cited around 1.5 times more often than
non-open-access papers. Similarly, papers with associated
open data were cited 9--50\% more than those
without\autocite{Piwowar:2013cc,McKiernan:2016iz}.
Vandewalle\autocite{Vandewalle:2012cl} showed papers in the image-processing
field receive up to three times the number of citations when source code is
made available.

\paragraph{Establish priority:}
Some researchers hesitate to embrace open science out of a fear of being ``scooped,''
where competitors will use some findings, software tools, or data made available
and then publish first. However, contrary to this belief, practicing open science
can actually prevent being scooped: releasing preprints can establish priority of
discoveries or techniques prior to the publication of a traditional peer-reviewed
journal article\autocite{Berg:2016gl,Strasser:2016fr}.

The peer-review and editorial process of such papers can take many months or years,
but journal articles are still necessary for research findings to be considered valid
(and for researchers to receive credit). Publishing a preprint of an article
publicly time-stamps the work, even as it undergoes peer review and possible
revision.

\paragraph{Encourages trust:}
Embracing openness in scientific research can help encourage other researchers
to trust published results, by giving the ability to inspect data or software.
Soergel estimated that 5--100\% of computational results given by software
may be incorrect or inaccurate\autocite{Soergel:2015ef}. While simply releasing
source code openly will not solve this problem, this is a necessary step towards
verification and reproducibility.

\paragraph{It's nice:}
In addition to the above benefits, sharing products of research openly is kind
to colleagues and the greater research community, as it prevents people from
wasting time repeating work unnecessarily.
Like many graduate students, I began working on my dissertation research by
attempting to reimplement another group's method and reproduce some of the
results in a paper. However, this group did not share any source code openly.
As a result, I wasted significant time dealing with minor implementation details
or inputs not discussed in their papers. This could have been avoided by sharing
the source code, which would have allowed me to more quickly move on to new work.
Graduate students and other researchers constantly face similar challenges that
could be avoided by greater openness in research.



\section{Openness increases societal impact of research}
\label{sec:impact}

Many published journal articles go unread, even in their topical domains.
One study of citation rates found that 27\% of papers published in the natural
sciences and engineering go uncited\autocite{Lariviere:2009}\footnote{Of course,
papers that are read may not be cited, and papers that are cited may not actually
be read.}
Those who do read most papers likely come from research institutions similar to
those of the authors, even if the findings could be impactful beyond these
confines, for example leading to policy changes or technological solutions
for the developing world. In part this is due to the challenging technical content,
jargon, and niche topics---but it is also due to the lack of access to the journals
where most research findings reside. (Making the content of these papers actually
understandable or digestible by most is another challenge.)

Considering the high and ever-increasing cost of scholarly journal subscriptions,
research results should not be limited to those with the means to purchase access.
By publishing articles in open-access journals or self-archiving (i.e., green open
access), researchers can ensure access for all more members of society, including
policymakers, funders, members of the media, entrepreneurs, and the general
public---or scientists and engineers in the Global South.

Furthermore, being more open with all outputs of research (e.g., papers, software,
data) could help improve the general public's perception and trust in
scientific research. Simply making research products available will not solve
all of these problems---for one, it will not sway those who strongly believe
ideas contrary to fact. However, ensuring everyone has access to the data researchers
generate and analyze, and the software tools on which we rely, could eliminate one
major barrier to trust in our findings\autocite{Grand:2012}.

\section{Open science and my research agenda}
\label{sec:agenda}

Over the past year and a half (September 2015--February 2017), as I initiated
and built my research group, I worked to crystallize my vision and plan for
performing research openly. The guiding principles I formed were heavily
inspired by the Reproducibility PI Manifesto of Lorena Barba\autocite{Barba:2012pi},
the Peer Reviewers' Openness Initiative\autocite{Morey150547}, and others.
However, in addition to these exemplars, my goals are also motivated by the desire
to work more openly than the norm in my field.

My field, which includes computational modeling of combustion and fluid dynamics,
chemical kinetics, and associated numerical methods, lacks reputable open-access
journals. As a graduate student, I began submitting my preprints of my publications
to the arXiv\footnote{\url{https://arxiv.org}}, and later archiving my
(otherwise non-accessible) conference papers on
Figshare\footnote{\url{https://figshare.com}}.

Since then, I have adopted the following policy for my group's research:
\begin{itemize}
    \item All written research products are made openly accessible, either through
    green or gold open-access avenues. Since my field generally lacks recognized,
    fully open-access journals, this objective is typically met by submitting
    preprints to services such as arXiv, engrXiv, or PeerJ Preprints, depending
    on the paper's topic. Conference papers, when not submitted to an open venue,
    are also made openly available through, e.g., Figshare. Where possible, all
    preprints will be released under the Creative Commons Attribution (CC BY)
    license.\footnote{\url{https://creativecommons.org/licenses/by/4.0/legalcode}}
    (I choose, in general, to not follow the hybrid gold open-access model by
    paying a non-fully open journal.)

    \item All new research software developed by my group will be done so openly
    (e.g., on GitHub) and released publicly under a permissive license such as
    the BSD 3-clause license. The Git version-control system will be used to
    track the history of software projects, and releases associated with
    publications or data will be archived (with DOIs) using
    Zenodo\footnote{\url{https://zenodo.org}}.
    In addition, we will make all efforts to describe as many implementation
    details as necessary to reproduce our work.

    \item All data generated through research, when serving as the basis for
    a publication, will be archived publicly and cited appropriately in our
    work. Figures and plotting scripts will be shared via the CC BY license
    and cited in our papers.
\end{itemize}
This policy will be implemented in part by incorporating its content into
funding proposals, for example in Data Management Plans.

In addition to these standards I established for my and my group's research, I
have also gotten involved with and helped initiate some community efforts to
encourage and reward open science practices.

First, after participating in the working group on software credit at the Third
Workshop on Sustainable Software for Science: Practice and Experiences
(WSSSPE3)\autocite{Katz:2016er}, I joined the FORCE11 Software Citation Working
Group as a cochair, and helped coauthor the resulting Software Citation
Principles\autocite{Smith:2016kt}. One purpose of standardizing software
citation is to help ensure authors\slash developers receive academic credit for
their work in releasing open research software. I also serve on the editorial
boards of the \textit{Journal of Open Research
Software} and \textit{Journal of Open Source Software}\footnote{\url{http://joss.theoj.org/};
\url{http://openresearchsoftware.metajnl.com/}},
which offer publication venues focused solely on open-source research software.

I have also collaborated with colleagues in engineering fields to improve
venues for open-access publications. I serve on the editorial board of
\textit{The Journal of Open Engineering},
and published an editorial discussing the importance of open publishing in
engineering\autocite{TJOE:editorial}. With some of the same collaborators, I
helped initiate engrXiv\footnote{\url{http://engrxiv.org}}, an open archive
for engineering publications, inspired by the arXiv.


\section{Challenges to performing open science}
\label{sec:challenges}

In my view, the challenges impeding greater adoption of open-science practices
are mainly institutional and cultural, rather than technical. General venues for sharing
and developing the products of research openly abound these days, with the availability
of services like arXiv, engrXiv, and PeerJ Preprints for ensuring open access of
publications; repositories like GitHub for developing (and version-controlling)
research software openly; and data and software archives like Zenodo and Figshare,
which practically have no file size limitations.\footnote{Zenodo currently accepts
datasets up to 50GB, but stores data in the CERN Data Center, along with 100PB of
physics data from the Large Hadron Collider (\url{https://zenodo.org/faq}).}
Of course, some technical problems remain: How do we make results of computational
science, particularly when it involves demanding high-performance computing resources,
truly reproducible? How can we cite software and data consistently, when
the version might change regularly?

Instead, cultural inertia and lack of institutional recognition\slash rewards pose
the most significant challenges to increased openness in science.

In my opinion, the biggest barrier to greater openness in research is ambivalence
or outright hostility in many research communities. Many academic researchers
either disagree on or are unaware of the importance (and benefits) of working
openly. Since they were not trained in doing this, e.g., during graduate school
or during postdoctoral training, they also may simply be unaware of how to
research openly, or the resources that are available to do so.
Furthermore, since most of their colleagues, collaborators,
and competitors do not practice open science, no pressure comes from the
research community to change. In addition, some communities do not support or
actively oppose activities such as submitting preprints.

This lack of pressure is related to the other major issue: lack of institutional
recognition and reward for open practices. In general, academic researchers will
work on what gets them credit for promotion and tenure---anything beyond that
requires strong intrinsic motivation, or external motivators from the research
community. At most institutions, promotion and tenure review includes some
judgement (whether explicit or implicit) of where faculty publish their work,
but many, ``high-impact''' traditional publication venues---particularly domain
journals---may not support, e.g., the posting of preprints.

\section{Recommendations for university leaders}
\label{sec:recommendations}

%Some funding agencies have begun to mandate that the researchers they fund share
%the products of their work openly.

The following list of recommendations are mostly targeted at changing criteria
for promotion and tenure, and performance reviews, to encourage faculty to
practice more open science:
\begin{itemize}
    \item Consider accessibility\slash openness of research products along
    with quantity and ``quality'' in promotion and tenure review.

    \item Recognize research products such as software and data, and their
    associated impacts (e.g., citations), as equal to traditional
    publications in scholarly impact.

    \item Reduce the importance of publishing in traditional venues for promotion
    and tenure, recognizing these may be barriers to openness.

    \item Support efforts to teach undergraduate and graduate students about
    open science skills, and those necessary to work with software and data, with
    the same enthusiasm that traditional lab courses receive.
\end{itemize}
Research communities that impede openness cannot be forced to change from the
outside. Instead, by making changes to institutional award systems, researchers
will be encouraged to improve their open practices, and thus evolve communities
from the inside.

%\bibliography{refs}
%\bibliographystyle{plainnat}
\printbibliography


\end{document}
